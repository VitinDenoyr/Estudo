%%% Newden Repository %%%
%%% By: Paulo Vitor Correia de Oliveira %%%
\documentclass[11pt, a4paper]{article}
\usepackage{denplate}

%%% ===== Metadata ===== %%%
\title{Repositório NewDen}
\author{Paulo Vitor Correia de Oliveira}
\date{}

\begin{document}
\maketitle

\void[-3.5]

\section{Algoritmos}

Neste documento serão abordadas desde técnicas e estruturas padrão de programação competitiva, até possivelmente entrar em tópicos um pouco mais avançados. Entraremos em cada um dos tópicos pertinentes a programação competitiva e exploraremos os principais tópicos relacionados a eles.

\section{Problemas}

Neste documento serão abordados problemas interessantes de múltiplas áreas do conhecimento. Em particular, podemos ver problemas que intersectam as áreas de todos os outros documentos, como programação competitiva, fornecendo ideias úteis que podem contribuir muito com o desenvolvimento do seu raciocínio.

\section{Desenvolvimento}

Neste documento serão abordados aspectos mais diretamente computacionais: sintaxes de linguagens, métodos sobre o desenvolvimento de algum software, dentre coisas similares. Não serão vistos temas amplos que podem ser transformados em uma seção completa.

\section{Cálculo}

Este documento cobre os conteúdos essenciais relacionados a cálculo e também análise, em geral, aqueles abordados no IMPA Tech. Entre os temas estão cálculo diferencial e integral, cálculo vetorial, cálculo multivariável, equações diferenciais ordinárias e análise 1.

\section{Probabilidade}

Neste documento, veremos os temas pertinentes a probabilidade e estatística, por exemplo o estudo de variáveis aleatórias, distribuições de probabilidade, inferência estatística, dentre outros temas da mesma vertente.

\section{Álgebra Linear}

Este documento aborda os temas principais no estudo de álgebra linear tradicional e também numérica. Veremos matrizes, transformações lineares, algoritmos clássicos implementados numericamente e também discussões sobre estabilidade deles.

\section{Teoria dos Jogos}

Este documento tem como objetivo um estudo mais profundo sobre teoria dos jogos. Veremos uma introdução às quatro grandes classes de jogos, que se diferenciam como estáticos e dinâmicos e também pela informação completa ou incompleta. Além disso, serão exploradas classes especiais de jogos, como os jogos combinatoriais.

\section{Programação Paralela}

Neste documento, estudaremos o que é programação paralela, para que ela serve e como podemos usar isso para acelerar tarefas e quão rápido podemos acelerar algo de fato. Tendo em vista que já temos uma ideia de como um programa opera em alto nível, veremos como podemos usar esse paradigma para facilitar ou melhorar tarefas que antes eram feitas de forma menos eficiente.
Neste documento, estudaremos o que é programação paralela, para que ela serve e como podemos usar isso para acelerar tarefas e quão rápido podemos acelerar algo de fato. Tendo em vista que já temos uma ideia de como um programa opera em alto nível, veremos como podemos usar esse paradigma para facilitar ou melhorar tarefas que antes eram feitas de forma menos eficiente.

\end{document}
